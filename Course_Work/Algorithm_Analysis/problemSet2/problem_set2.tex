
% Default to the notebook output style

    


% Inherit from the specified cell style.




    
\documentclass[11pt]{article}

    
    
    \usepackage[T1]{fontenc}
    % Nicer default font (+ math font) than Computer Modern for most use cases
    \usepackage{mathpazo}

    % Basic figure setup, for now with no caption control since it's done
    % automatically by Pandoc (which extracts ![](path) syntax from Markdown).
    \usepackage{graphicx}
    % We will generate all images so they have a width \maxwidth. This means
    % that they will get their normal width if they fit onto the page, but
    % are scaled down if they would overflow the margins.
    \makeatletter
    \def\maxwidth{\ifdim\Gin@nat@width>\linewidth\linewidth
    \else\Gin@nat@width\fi}
    \makeatother
    \let\Oldincludegraphics\includegraphics
    % Set max figure width to be 80% of text width, for now hardcoded.
    \renewcommand{\includegraphics}[1]{\Oldincludegraphics[width=.8\maxwidth]{#1}}
    % Ensure that by default, figures have no caption (until we provide a
    % proper Figure object with a Caption API and a way to capture that
    % in the conversion process - todo).
    \usepackage{caption}
    \DeclareCaptionLabelFormat{nolabel}{}
    \captionsetup{labelformat=nolabel}

    \usepackage{adjustbox} % Used to constrain images to a maximum size 
    \usepackage{xcolor} % Allow colors to be defined
    \usepackage{enumerate} % Needed for markdown enumerations to work
    \usepackage{geometry} % Used to adjust the document margins
    \usepackage{amsmath} % Equations
    \usepackage{amssymb} % Equations
    \usepackage{textcomp} % defines textquotesingle
    % Hack from http://tex.stackexchange.com/a/47451/13684:
    \AtBeginDocument{%
        \def\PYZsq{\textquotesingle}% Upright quotes in Pygmentized code
    }
    \usepackage{upquote} % Upright quotes for verbatim code
    \usepackage{eurosym} % defines \euro
    \usepackage[mathletters]{ucs} % Extended unicode (utf-8) support
    \usepackage[utf8x]{inputenc} % Allow utf-8 characters in the tex document
    \usepackage{fancyvrb} % verbatim replacement that allows latex
    \usepackage{grffile} % extends the file name processing of package graphics 
                         % to support a larger range 
    % The hyperref package gives us a pdf with properly built
    % internal navigation ('pdf bookmarks' for the table of contents,
    % internal cross-reference links, web links for URLs, etc.)
    \usepackage{hyperref}
    \usepackage{longtable} % longtable support required by pandoc >1.10
    \usepackage{booktabs}  % table support for pandoc > 1.12.2
    \usepackage[inline]{enumitem} % IRkernel/repr support (it uses the enumerate* environment)
    \usepackage[normalem]{ulem} % ulem is needed to support strikethroughs (\sout)
                                % normalem makes italics be italics, not underlines
    \usepackage{mathrsfs}
    

    
    
    % Colors for the hyperref package
    \definecolor{urlcolor}{rgb}{0,.145,.698}
    \definecolor{linkcolor}{rgb}{.71,0.21,0.01}
    \definecolor{citecolor}{rgb}{.12,.54,.11}

    % ANSI colors
    \definecolor{ansi-black}{HTML}{3E424D}
    \definecolor{ansi-black-intense}{HTML}{282C36}
    \definecolor{ansi-red}{HTML}{E75C58}
    \definecolor{ansi-red-intense}{HTML}{B22B31}
    \definecolor{ansi-green}{HTML}{00A250}
    \definecolor{ansi-green-intense}{HTML}{007427}
    \definecolor{ansi-yellow}{HTML}{DDB62B}
    \definecolor{ansi-yellow-intense}{HTML}{B27D12}
    \definecolor{ansi-blue}{HTML}{208FFB}
    \definecolor{ansi-blue-intense}{HTML}{0065CA}
    \definecolor{ansi-magenta}{HTML}{D160C4}
    \definecolor{ansi-magenta-intense}{HTML}{A03196}
    \definecolor{ansi-cyan}{HTML}{60C6C8}
    \definecolor{ansi-cyan-intense}{HTML}{258F8F}
    \definecolor{ansi-white}{HTML}{C5C1B4}
    \definecolor{ansi-white-intense}{HTML}{A1A6B2}
    \definecolor{ansi-default-inverse-fg}{HTML}{FFFFFF}
    \definecolor{ansi-default-inverse-bg}{HTML}{000000}

    % commands and environments needed by pandoc snippets
    % extracted from the output of `pandoc -s`
    \providecommand{\tightlist}{%
      \setlength{\itemsep}{0pt}\setlength{\parskip}{0pt}}
    \DefineVerbatimEnvironment{Highlighting}{Verbatim}{commandchars=\\\{\}}
    % Add ',fontsize=\small' for more characters per line
    \newenvironment{Shaded}{}{}
    \newcommand{\KeywordTok}[1]{\textcolor[rgb]{0.00,0.44,0.13}{\textbf{{#1}}}}
    \newcommand{\DataTypeTok}[1]{\textcolor[rgb]{0.56,0.13,0.00}{{#1}}}
    \newcommand{\DecValTok}[1]{\textcolor[rgb]{0.25,0.63,0.44}{{#1}}}
    \newcommand{\BaseNTok}[1]{\textcolor[rgb]{0.25,0.63,0.44}{{#1}}}
    \newcommand{\FloatTok}[1]{\textcolor[rgb]{0.25,0.63,0.44}{{#1}}}
    \newcommand{\CharTok}[1]{\textcolor[rgb]{0.25,0.44,0.63}{{#1}}}
    \newcommand{\StringTok}[1]{\textcolor[rgb]{0.25,0.44,0.63}{{#1}}}
    \newcommand{\CommentTok}[1]{\textcolor[rgb]{0.38,0.63,0.69}{\textit{{#1}}}}
    \newcommand{\OtherTok}[1]{\textcolor[rgb]{0.00,0.44,0.13}{{#1}}}
    \newcommand{\AlertTok}[1]{\textcolor[rgb]{1.00,0.00,0.00}{\textbf{{#1}}}}
    \newcommand{\FunctionTok}[1]{\textcolor[rgb]{0.02,0.16,0.49}{{#1}}}
    \newcommand{\RegionMarkerTok}[1]{{#1}}
    \newcommand{\ErrorTok}[1]{\textcolor[rgb]{1.00,0.00,0.00}{\textbf{{#1}}}}
    \newcommand{\NormalTok}[1]{{#1}}
    
    % Additional commands for more recent versions of Pandoc
    \newcommand{\ConstantTok}[1]{\textcolor[rgb]{0.53,0.00,0.00}{{#1}}}
    \newcommand{\SpecialCharTok}[1]{\textcolor[rgb]{0.25,0.44,0.63}{{#1}}}
    \newcommand{\VerbatimStringTok}[1]{\textcolor[rgb]{0.25,0.44,0.63}{{#1}}}
    \newcommand{\SpecialStringTok}[1]{\textcolor[rgb]{0.73,0.40,0.53}{{#1}}}
    \newcommand{\ImportTok}[1]{{#1}}
    \newcommand{\DocumentationTok}[1]{\textcolor[rgb]{0.73,0.13,0.13}{\textit{{#1}}}}
    \newcommand{\AnnotationTok}[1]{\textcolor[rgb]{0.38,0.63,0.69}{\textbf{\textit{{#1}}}}}
    \newcommand{\CommentVarTok}[1]{\textcolor[rgb]{0.38,0.63,0.69}{\textbf{\textit{{#1}}}}}
    \newcommand{\VariableTok}[1]{\textcolor[rgb]{0.10,0.09,0.49}{{#1}}}
    \newcommand{\ControlFlowTok}[1]{\textcolor[rgb]{0.00,0.44,0.13}{\textbf{{#1}}}}
    \newcommand{\OperatorTok}[1]{\textcolor[rgb]{0.40,0.40,0.40}{{#1}}}
    \newcommand{\BuiltInTok}[1]{{#1}}
    \newcommand{\ExtensionTok}[1]{{#1}}
    \newcommand{\PreprocessorTok}[1]{\textcolor[rgb]{0.74,0.48,0.00}{{#1}}}
    \newcommand{\AttributeTok}[1]{\textcolor[rgb]{0.49,0.56,0.16}{{#1}}}
    \newcommand{\InformationTok}[1]{\textcolor[rgb]{0.38,0.63,0.69}{\textbf{\textit{{#1}}}}}
    \newcommand{\WarningTok}[1]{\textcolor[rgb]{0.38,0.63,0.69}{\textbf{\textit{{#1}}}}}
    
    
    % Define a nice break command that doesn't care if a line doesn't already
    % exist.
    \def\br{\hspace*{\fill} \\* }
    % Math Jax compatibility definitions
    \def\gt{>}
    \def\lt{<}
    \let\Oldtex\TeX
    \let\Oldlatex\LaTeX
    \renewcommand{\TeX}{\textrm{\Oldtex}}
    \renewcommand{\LaTeX}{\textrm{\Oldlatex}}
    % Document parameters
    % Document title
    \title{problem\_set2}
    
    
    
    
    

    % Pygments definitions
    
\makeatletter
\def\PY@reset{\let\PY@it=\relax \let\PY@bf=\relax%
    \let\PY@ul=\relax \let\PY@tc=\relax%
    \let\PY@bc=\relax \let\PY@ff=\relax}
\def\PY@tok#1{\csname PY@tok@#1\endcsname}
\def\PY@toks#1+{\ifx\relax#1\empty\else%
    \PY@tok{#1}\expandafter\PY@toks\fi}
\def\PY@do#1{\PY@bc{\PY@tc{\PY@ul{%
    \PY@it{\PY@bf{\PY@ff{#1}}}}}}}
\def\PY#1#2{\PY@reset\PY@toks#1+\relax+\PY@do{#2}}

\expandafter\def\csname PY@tok@w\endcsname{\def\PY@tc##1{\textcolor[rgb]{0.73,0.73,0.73}{##1}}}
\expandafter\def\csname PY@tok@c\endcsname{\let\PY@it=\textit\def\PY@tc##1{\textcolor[rgb]{0.25,0.50,0.50}{##1}}}
\expandafter\def\csname PY@tok@cp\endcsname{\def\PY@tc##1{\textcolor[rgb]{0.74,0.48,0.00}{##1}}}
\expandafter\def\csname PY@tok@k\endcsname{\let\PY@bf=\textbf\def\PY@tc##1{\textcolor[rgb]{0.00,0.50,0.00}{##1}}}
\expandafter\def\csname PY@tok@kp\endcsname{\def\PY@tc##1{\textcolor[rgb]{0.00,0.50,0.00}{##1}}}
\expandafter\def\csname PY@tok@kt\endcsname{\def\PY@tc##1{\textcolor[rgb]{0.69,0.00,0.25}{##1}}}
\expandafter\def\csname PY@tok@o\endcsname{\def\PY@tc##1{\textcolor[rgb]{0.40,0.40,0.40}{##1}}}
\expandafter\def\csname PY@tok@ow\endcsname{\let\PY@bf=\textbf\def\PY@tc##1{\textcolor[rgb]{0.67,0.13,1.00}{##1}}}
\expandafter\def\csname PY@tok@nb\endcsname{\def\PY@tc##1{\textcolor[rgb]{0.00,0.50,0.00}{##1}}}
\expandafter\def\csname PY@tok@nf\endcsname{\def\PY@tc##1{\textcolor[rgb]{0.00,0.00,1.00}{##1}}}
\expandafter\def\csname PY@tok@nc\endcsname{\let\PY@bf=\textbf\def\PY@tc##1{\textcolor[rgb]{0.00,0.00,1.00}{##1}}}
\expandafter\def\csname PY@tok@nn\endcsname{\let\PY@bf=\textbf\def\PY@tc##1{\textcolor[rgb]{0.00,0.00,1.00}{##1}}}
\expandafter\def\csname PY@tok@ne\endcsname{\let\PY@bf=\textbf\def\PY@tc##1{\textcolor[rgb]{0.82,0.25,0.23}{##1}}}
\expandafter\def\csname PY@tok@nv\endcsname{\def\PY@tc##1{\textcolor[rgb]{0.10,0.09,0.49}{##1}}}
\expandafter\def\csname PY@tok@no\endcsname{\def\PY@tc##1{\textcolor[rgb]{0.53,0.00,0.00}{##1}}}
\expandafter\def\csname PY@tok@nl\endcsname{\def\PY@tc##1{\textcolor[rgb]{0.63,0.63,0.00}{##1}}}
\expandafter\def\csname PY@tok@ni\endcsname{\let\PY@bf=\textbf\def\PY@tc##1{\textcolor[rgb]{0.60,0.60,0.60}{##1}}}
\expandafter\def\csname PY@tok@na\endcsname{\def\PY@tc##1{\textcolor[rgb]{0.49,0.56,0.16}{##1}}}
\expandafter\def\csname PY@tok@nt\endcsname{\let\PY@bf=\textbf\def\PY@tc##1{\textcolor[rgb]{0.00,0.50,0.00}{##1}}}
\expandafter\def\csname PY@tok@nd\endcsname{\def\PY@tc##1{\textcolor[rgb]{0.67,0.13,1.00}{##1}}}
\expandafter\def\csname PY@tok@s\endcsname{\def\PY@tc##1{\textcolor[rgb]{0.73,0.13,0.13}{##1}}}
\expandafter\def\csname PY@tok@sd\endcsname{\let\PY@it=\textit\def\PY@tc##1{\textcolor[rgb]{0.73,0.13,0.13}{##1}}}
\expandafter\def\csname PY@tok@si\endcsname{\let\PY@bf=\textbf\def\PY@tc##1{\textcolor[rgb]{0.73,0.40,0.53}{##1}}}
\expandafter\def\csname PY@tok@se\endcsname{\let\PY@bf=\textbf\def\PY@tc##1{\textcolor[rgb]{0.73,0.40,0.13}{##1}}}
\expandafter\def\csname PY@tok@sr\endcsname{\def\PY@tc##1{\textcolor[rgb]{0.73,0.40,0.53}{##1}}}
\expandafter\def\csname PY@tok@ss\endcsname{\def\PY@tc##1{\textcolor[rgb]{0.10,0.09,0.49}{##1}}}
\expandafter\def\csname PY@tok@sx\endcsname{\def\PY@tc##1{\textcolor[rgb]{0.00,0.50,0.00}{##1}}}
\expandafter\def\csname PY@tok@m\endcsname{\def\PY@tc##1{\textcolor[rgb]{0.40,0.40,0.40}{##1}}}
\expandafter\def\csname PY@tok@gh\endcsname{\let\PY@bf=\textbf\def\PY@tc##1{\textcolor[rgb]{0.00,0.00,0.50}{##1}}}
\expandafter\def\csname PY@tok@gu\endcsname{\let\PY@bf=\textbf\def\PY@tc##1{\textcolor[rgb]{0.50,0.00,0.50}{##1}}}
\expandafter\def\csname PY@tok@gd\endcsname{\def\PY@tc##1{\textcolor[rgb]{0.63,0.00,0.00}{##1}}}
\expandafter\def\csname PY@tok@gi\endcsname{\def\PY@tc##1{\textcolor[rgb]{0.00,0.63,0.00}{##1}}}
\expandafter\def\csname PY@tok@gr\endcsname{\def\PY@tc##1{\textcolor[rgb]{1.00,0.00,0.00}{##1}}}
\expandafter\def\csname PY@tok@ge\endcsname{\let\PY@it=\textit}
\expandafter\def\csname PY@tok@gs\endcsname{\let\PY@bf=\textbf}
\expandafter\def\csname PY@tok@gp\endcsname{\let\PY@bf=\textbf\def\PY@tc##1{\textcolor[rgb]{0.00,0.00,0.50}{##1}}}
\expandafter\def\csname PY@tok@go\endcsname{\def\PY@tc##1{\textcolor[rgb]{0.53,0.53,0.53}{##1}}}
\expandafter\def\csname PY@tok@gt\endcsname{\def\PY@tc##1{\textcolor[rgb]{0.00,0.27,0.87}{##1}}}
\expandafter\def\csname PY@tok@err\endcsname{\def\PY@bc##1{\setlength{\fboxsep}{0pt}\fcolorbox[rgb]{1.00,0.00,0.00}{1,1,1}{\strut ##1}}}
\expandafter\def\csname PY@tok@kc\endcsname{\let\PY@bf=\textbf\def\PY@tc##1{\textcolor[rgb]{0.00,0.50,0.00}{##1}}}
\expandafter\def\csname PY@tok@kd\endcsname{\let\PY@bf=\textbf\def\PY@tc##1{\textcolor[rgb]{0.00,0.50,0.00}{##1}}}
\expandafter\def\csname PY@tok@kn\endcsname{\let\PY@bf=\textbf\def\PY@tc##1{\textcolor[rgb]{0.00,0.50,0.00}{##1}}}
\expandafter\def\csname PY@tok@kr\endcsname{\let\PY@bf=\textbf\def\PY@tc##1{\textcolor[rgb]{0.00,0.50,0.00}{##1}}}
\expandafter\def\csname PY@tok@bp\endcsname{\def\PY@tc##1{\textcolor[rgb]{0.00,0.50,0.00}{##1}}}
\expandafter\def\csname PY@tok@fm\endcsname{\def\PY@tc##1{\textcolor[rgb]{0.00,0.00,1.00}{##1}}}
\expandafter\def\csname PY@tok@vc\endcsname{\def\PY@tc##1{\textcolor[rgb]{0.10,0.09,0.49}{##1}}}
\expandafter\def\csname PY@tok@vg\endcsname{\def\PY@tc##1{\textcolor[rgb]{0.10,0.09,0.49}{##1}}}
\expandafter\def\csname PY@tok@vi\endcsname{\def\PY@tc##1{\textcolor[rgb]{0.10,0.09,0.49}{##1}}}
\expandafter\def\csname PY@tok@vm\endcsname{\def\PY@tc##1{\textcolor[rgb]{0.10,0.09,0.49}{##1}}}
\expandafter\def\csname PY@tok@sa\endcsname{\def\PY@tc##1{\textcolor[rgb]{0.73,0.13,0.13}{##1}}}
\expandafter\def\csname PY@tok@sb\endcsname{\def\PY@tc##1{\textcolor[rgb]{0.73,0.13,0.13}{##1}}}
\expandafter\def\csname PY@tok@sc\endcsname{\def\PY@tc##1{\textcolor[rgb]{0.73,0.13,0.13}{##1}}}
\expandafter\def\csname PY@tok@dl\endcsname{\def\PY@tc##1{\textcolor[rgb]{0.73,0.13,0.13}{##1}}}
\expandafter\def\csname PY@tok@s2\endcsname{\def\PY@tc##1{\textcolor[rgb]{0.73,0.13,0.13}{##1}}}
\expandafter\def\csname PY@tok@sh\endcsname{\def\PY@tc##1{\textcolor[rgb]{0.73,0.13,0.13}{##1}}}
\expandafter\def\csname PY@tok@s1\endcsname{\def\PY@tc##1{\textcolor[rgb]{0.73,0.13,0.13}{##1}}}
\expandafter\def\csname PY@tok@mb\endcsname{\def\PY@tc##1{\textcolor[rgb]{0.40,0.40,0.40}{##1}}}
\expandafter\def\csname PY@tok@mf\endcsname{\def\PY@tc##1{\textcolor[rgb]{0.40,0.40,0.40}{##1}}}
\expandafter\def\csname PY@tok@mh\endcsname{\def\PY@tc##1{\textcolor[rgb]{0.40,0.40,0.40}{##1}}}
\expandafter\def\csname PY@tok@mi\endcsname{\def\PY@tc##1{\textcolor[rgb]{0.40,0.40,0.40}{##1}}}
\expandafter\def\csname PY@tok@il\endcsname{\def\PY@tc##1{\textcolor[rgb]{0.40,0.40,0.40}{##1}}}
\expandafter\def\csname PY@tok@mo\endcsname{\def\PY@tc##1{\textcolor[rgb]{0.40,0.40,0.40}{##1}}}
\expandafter\def\csname PY@tok@ch\endcsname{\let\PY@it=\textit\def\PY@tc##1{\textcolor[rgb]{0.25,0.50,0.50}{##1}}}
\expandafter\def\csname PY@tok@cm\endcsname{\let\PY@it=\textit\def\PY@tc##1{\textcolor[rgb]{0.25,0.50,0.50}{##1}}}
\expandafter\def\csname PY@tok@cpf\endcsname{\let\PY@it=\textit\def\PY@tc##1{\textcolor[rgb]{0.25,0.50,0.50}{##1}}}
\expandafter\def\csname PY@tok@c1\endcsname{\let\PY@it=\textit\def\PY@tc##1{\textcolor[rgb]{0.25,0.50,0.50}{##1}}}
\expandafter\def\csname PY@tok@cs\endcsname{\let\PY@it=\textit\def\PY@tc##1{\textcolor[rgb]{0.25,0.50,0.50}{##1}}}

\def\PYZbs{\char`\\}
\def\PYZus{\char`\_}
\def\PYZob{\char`\{}
\def\PYZcb{\char`\}}
\def\PYZca{\char`\^}
\def\PYZam{\char`\&}
\def\PYZlt{\char`\<}
\def\PYZgt{\char`\>}
\def\PYZsh{\char`\#}
\def\PYZpc{\char`\%}
\def\PYZdl{\char`\$}
\def\PYZhy{\char`\-}
\def\PYZsq{\char`\'}
\def\PYZdq{\char`\"}
\def\PYZti{\char`\~}
% for compatibility with earlier versions
\def\PYZat{@}
\def\PYZlb{[}
\def\PYZrb{]}
\makeatother


    % Exact colors from NB
    \definecolor{incolor}{rgb}{0.0, 0.0, 0.5}
    \definecolor{outcolor}{rgb}{0.545, 0.0, 0.0}



    
    % Prevent overflowing lines due to hard-to-break entities
    \sloppy 
    % Setup hyperref package
    \hypersetup{
      breaklinks=true,  % so long urls are correctly broken across lines
      colorlinks=true,
      urlcolor=urlcolor,
      linkcolor=linkcolor,
      citecolor=citecolor,
      }
    % Slightly bigger margins than the latex defaults
    
    \geometry{verbose,tmargin=1in,bmargin=1in,lmargin=1in,rmargin=1in}
    
    

    \begin{document}
    
    
    \maketitle
    
    

    
    \begin{enumerate}
\def\labelenumi{\arabic{enumi}.}
\tightlist
\item
  Explain how to solve the following two problems using heaps. First,
  give an O(nlogk) algorithm to merge k sorted lists with n total
  elements into one sorted list. `
\end{enumerate}

    Given k sorted input arrays \(A_{1}\), \(A_{2}\), \ldots, \(A_{k}\),
where the sum of the length of the two arrays is \(n\). Create a heap
\(B\) of size \(k\), where each element in \(B\) is a single element
from each of the k input ararys. Furthermore, each element is taken from
the last most entry of each input array. This ensures that \(B\) will
always be made up of the largest possible values out of all input
arrays. Call Build\_Max\_Heap on \(B\) to ensure that the heap
statisfies the max heap property. This will run in \(O(k)\) time. Next
remove the root node from \(B\) and insert to results array. Add new
number from the same array that the element in the root node came from.
This ensures that each element in \(B\) comes from each input array,
which ensures that the max most value from each array is in the heap.
This is necessary for proper sorting. This process is repeated until the
heap becomes empty, and each element that is inserted into results is
inserted backwards from n, n-1, n-2, \ldots, 1. When an input array
becomes empty, negative infinity is inserted into \(B\), and the
algorithm continues as normal.

Pseudocode:\\
Input: k arrays \([A_{1}\), \(A_{2}\), \ldots, \(A_{k}]\) with total
length equal to \(n\).\\
Output: Single sorted array.

results \(\leftarrow [ ]\) // Initialize results into an empty array
that will contain the sorted elements.\\
lists \(\leftarrow [A_{1}, A_{2}, ..., A_{k}]\) B \(\leftarrow [a_{1}\),
\(a_{2}\), \ldots, \(a_{k}]\) // B is formed by taking the last element
from \([A_{1}\), \(A_{2}\), \ldots, \(A_{k}]\). It can be shown that
this would take \(O(1)\) time by using indexing.\\
Build\_max\_heap(\(B\)) // Runs in \(O(k)\) time.

while \(B \neq \emptyset\) \{\\
\(i, v \leftarrow\) Extract\_max(\(B\)) // runs in \(O(logk)\) time.
Extract\_max maintains the max heap property. \(i\) is the index of
lists, which corresponds to the input array that the value \(v\) came
from.

new\_element \(\leftarrow\)
lists{[}\(i\){]}{[}length(lists{[}\(i\){]}){]} // Runs in O(1) time.
This indexes for the last element in the same array as \(v\).

results{[}\(n\){]} \(\leftarrow v\) // adds the value to the end of
results.

\(n=n-1\)

if length(lists{[}\(i\){]}) = 0 \{\\
insert(\(B\), \(-\infty\)) // insert also maintains the max heap
property. Runtime is O(logk).\\
\}\\
else\{\\
insert(\(B\), new\_element)\\
\}\}

    The while loop runs until \(B\) is empty, which means it will run \(n\)
times because there are total n elements. That means that the total
runtime is \(O(k) + nO(logk) + nO(1) + nO(logk) = O(k+nlogk+n)\).
Clearly, \(nO(logk)\) dominates. Therefore, runtime is \(O(nlogk)\).

    Proof of correctness by loop invariant:

The \(i^{th}\) element to insert into result is always the root node in
heap \(B\) in step \(i\) because that has the highest value in the heap.
Therefore, the results array is correctly sorted when every element is
inserted backwards from index \(n\) down to index \(1\). At a given
instance, \(B\) always contain the max values from each input array.
Therefore, the root of \(B\) is the max value possible out of all the
input arrays, not including the values already in results.

Initialization: Show invariant is true before loop started.

Before the \(1^{st}\) iteration of the while loop, \(B\) always contain
the max values from each input list. Therefore, the root of \(B\) is the
max value, which is the correct element to insert into results.
Therefore, the algorithm will correctly sort this value.

Maintenance: Show it is true after an iteration.

After \(1^{st}\) iteration, the max element is inserted into results.
This means \(B\) contains the \(2^{nd}\) highest value possible, and it
is at the root node. Since algorithm removes element from root to add to
results, this element will be correctly sorted into results.

Therefore after \(i^{th}\) iterations, the \(i^{th}\) element is
correctly inserted into results{[}n\ldots i\ldots1{]}.

Termination:

The code will terminate when \(B \neq \emptyset\), which means input
arrays are now empty. Therefore, all elements must be in results. Since
algorithm correctly sorts at each iteration, the final output must be
correctly sorted. QED.

    \begin{enumerate}
\def\labelenumi{\arabic{enumi}.}
\tightlist
\item
  Second, say that a list of numbers is k-close to sorted if each number
  in the list is less than k positions from its actual place in the
  sorted order. (Hence, a list that is 1-close to sorted is actually
  sorted.) Give an O(nlogk) algorithm for sorting a list of n numbers
  that is k-close to sorted.
\end{enumerate}

    Given an input array \(A\) of length \(n\) with numbers that are k-close
to sorted. For every value that is k-close to sorted, create a binary
heap, \(B\), of size \(k+1\). \(B\) is made up of the last \(k+1\)
elements from A. This ensure that \(B\) will always contain the max
value from \(A\). Initialize an empty array called results. Call
Build\_Max\_Heap on B once. This runs in O(k). Now remove the root node
from \(B\) and insert it into results array. Remove the value in the
last position of \(A\) and insert it into \(B\). This is repeated n
times, and at each iteration, the value is inserted into results array
going backwards, from the last index of results to the first.

Pseudocode:

Input: \(A\), unsorted and k-close to sorted. \(A\) is length \(n\).
\(A[1, ..., n]\).\\
Output: return a sorted array called results.

results \(\leftarrow [ ]\) // Initialize results into an empty array
that will contain the sorted elements.\\
B \(\leftarrow A[k-1, k-2, ..., n]\) // B is formed by taking the last
\(k+1\) elements from A. It can be shown that this would take \(O(1)\)
time by using indexing.\\
Build\_max\_heap(\(B\)) // Runs in \(O(k)\) time.

while \(B \neq \emptyset\) \{\\
\(v \leftarrow\) Extract\_max(\(B\)) // runs in \(O(logk)\) time.
Extract\_max maintains the max heap property. \(i\) is the index of
lists, which corresponds to the input array that the value \(v\) came
from.

results{[}\(n\){]} \(\leftarrow v\) // adds the value to the end of
results.

\(n=n-1\)

insert(\(B\), \(A[n]\)) // insert also maintains the max heap property.
Runtime is O(logk).\\
\}

    The while loop runs n times. Therefore runtime is O(nlogk) + O(n) +
O(k). The dominant term is O(nlogk); therefore, the overall runtime is
O(nlogk).

    Proof of correctness by loop invariant:

At \(i^{th}\) step, the max value of \(B\) will be the correct element
to insert to results going from last index to first index of the array.
In other words, the \(i^{th}\) element to be inserted into results is
found in \(B\) in the \(i^{th}\) step. This implies that results will be
correctly sorted when the \(i^{th}\) element is inserted into results
from index \(n ... i\).

Initialization: Show invariant is true before loop started.

Before the first step, \(B\) contains the max value in \(A\) because the
max value must be \(k\) positions away from correct sorted position, and
\(B\) contains \(k+1\) elements starting with values from the end of
\(A\). This ensures that \(B\) will include the max value. Therefore,
the correct value will be inserted into results, which is found in the
root of \(B\). The root of \(B\) is the max most value.

Maintenance: Show it is true after an iteration.

After first iteration, the max most value is found in results. Because
\(B\) always contain the current max value of \(A\), we can be sure that
the correct value will be inserted into results by the algorithm.

Therefore, after \(i^{th}\) iteration the \(i^{th}\) element will be
correctly inserted to results.

Termination: Show termination of loop results in desired outcome.

It follows that after \(n\) iterations, loop terminates and the results
array contains all the sorted elements in correct order. QED.

    \begin{enumerate}
\def\labelenumi{\arabic{enumi}.}
\setcounter{enumi}{1}
\tightlist
\item
  Consider an algorithm for integer multiplication of two n-digit
  numbers where each number is split into 3 parts, each with n/3 digits.
  Design and explain such an algorithm, similar to the integer
  multiplication algorithm (karatsuba's algo) presented in class. Your
  algorithm should describe how to multiply the two integers using only
  6 multiplication on the smaller parts instead of the straight forward
  9.
\end{enumerate}

    Given 2 integers \(X\) and \(Y\). Each integer has length n and n is
divisible by 3. Let \(X = a10^{2n/3} + b10^{n/3} + c\) and
\(Y = d10^{2n/3} + e10^{n/3} + f\). Then the multi algorithm will make 6
recursive calls on inputs of size \(n/3\). The base case is when the two
inputs are length 1. At this recursive level, the algorithm will do
simple integer multiplication because the input is small enough where
the runtime will be \(O(1)\) for this step. Finally, the algorithm will
use the results of the 6 recursive calls to calculate the product of
\(X\) and \(Y\).

Pseudocode:

multi(X,Y)\{\\
Input: 2 positive integers \(X\) and \(Y\) where the length of each
integer is n.~Assume \(n\) is divisible by 3.\\
Output: Returns \(X \cdot Y\).

if \(X\) or \(Y\) are length 1\{\\
return product of \(X\) and \(Y\) // Integer multiplication on
sufficiently small numbers has runtime O(1) time.\\
\}

\(X = a10^{2n/3} + b10^{n/3} + c\)\\
\(Y = d10^{2n/3} + e10^{n/3} + f\)

temp1 \(\leftarrow\) multi(\(a\),\(d\))\\
temp2 \(\leftarrow\) multi(\(a+b\), \(e+d\))\\
temp3 \(\leftarrow\) multi(\(b\), \(e\))\\
temp4 \(\leftarrow\) multi(\(c+b\), \(e+f\))\\
temp5 \(\leftarrow\) multi(\(c\), \(f\))\\
temp6 \(\leftarrow\) multi(\(f+d\), \(a+c\))

return
\((temp1)10^{4n/3} + [temp2 - temp1 - temp3]10^{n} + [temp4 - temp5 - temp3]10^{n/3} + [temp6 - temp5 - temp3]10^{2n/3} + (temp5)\)
\}

    Proof of correctness by induction:

Input: 2 positive integers \(X\) and \(Y\) of length \(n\).\\
Output: Product of \(X\) and \(Y\).

Base case:\\
Suppose the length of \(X\) and \(Y\) is 1. Then \(X \cdot Y\) is
returned, which is the desired outcome.

Inductive Case:

Assume \(n\) is divisible by 3 and n is length of \(X\) and \(Y\). Let
\(X = a10^{2n/3} + b10^{n/3} + c\) and
\(Y = d10^{2n/3} + e10^{n/3} + f\). Assume that algorithm returns
correct integer multiplication on input of size smaller than \(n\).
Suppose we call algorithm on inputs of \(X\) and \(Y\) with length
\(n\). Therefore, the 6 recursive calls made by the algorithm will
return the correct product because the recursive calls are on inputs of
size \(n/3\) and the inductive hypothesis guarentees that this will
return correctly. Since it is a fact that
\(X \cdot Y = (ad)10^{4n/3} + (ae+db)10^{n} + (fa+be+dc)10^{2n/3} + (ce+bf)10^{n/3} + (cf)\)
and
\((temp1)10^{4n/3} + [temp2 - temp1 - temp3]10^{n} + [temp4 - temp5 - temp3]10^{n/3} + [temp6 - temp5 - temp3]10^{2n/3} + (temp5)\)
is equal to this, the final return call by the algorithm returns the
correct product. QED

    \begin{enumerate}
\def\labelenumi{\arabic{enumi}.}
\setcounter{enumi}{1}
\tightlist
\item
  Determine the asymptotic running time of your algorithm. Would you
  rather split it into two parts( with 3 multiplications on the smaller
  parts) as in karatsuba's algorithm?
\end{enumerate}

    There are 6 recursive calls on input of \(n/3\). Adding 2 n-digit
numbers has \(O(n)\) runtime. Multiplication by powers of 10 is O(n)
runtime. Therefore, the post processing is \(O(n) + O(n) = O(n)\). Let
\(T(n)\) represent the number of steps. Then,
\(T(n) = 6T(n/3) + O(n) = 6T(n/3) + cn\). Let \(a=6\), \(b=3\), \(k=1\).
Since \(6>3\), it follows from the master theorem that the runtime is
upper bounded by \(O(n^{log_{3}6}) = O(n^{1.63})\). Since karatsuba's
algorithm runs in \(O(n^{1.58})\), karatsuba's algorithm is slightly
faster, so it would be better to split it into two parts.

    \begin{enumerate}
\def\labelenumi{\arabic{enumi}.}
\setcounter{enumi}{1}
\tightlist
\item
  Suppose you could use only 5 multiplications instead of 6. Then
  determine the asymptotic running time of such an algorithm. In this
  case, would you rather split it into 2 parts or 3 parts?
\end{enumerate}

    \(T(n) = 5T(n/3) + O(n) = 6T(n/3) + cn\). Let \(a=5\), \(b=3\), \(k=1\).
Since \(5 > 3\), the runtime of this algorithm would be
\(O(n^{log_{3}5}) = O(n^{1.46})\). In this case, splitting into 3 parts
would be faster than karatsuba's algorithm.

    \begin{enumerate}
\def\labelenumi{\arabic{enumi}.}
\setcounter{enumi}{2}
\tightlist
\item
  An inversion in an array A{[}1, \ldots, n{]} is a pair of indices
  (i,j) such that i \textless{} j and A{[}i{]} \textgreater{} A{[}j{]}.
  The number of inversions in an n-element array is between 0 (if array
  is sorted) and n(n-1)/2 (if an array is sorted backwards). Describe
  and analyze an algorithm to count the n umber of inversions in an
  n-element array in O(nlogn) time. Modify mergesort.
\end{enumerate}

    Given an array \(A[1, 2, ..., n]\) and \(n\) is the length. The
algorithm does a recursive call on the first half of \(A\) and another
recursive call on the second half of \(A\). These recursive calls return
a value for the number of inversions and the sorted array. The merge
subroutine is called to merge the two halves of \(A\) into 1 sorted
array. At the same time, the number of inversions between the two halves
are calculated. The algorithm then returns the sum of all inversions
from the 3 parts.

Pseudocode:

Input is array \(A[1, 2, ..., n]\). The length of \(A\) is \(n\). Output
is number of inversions.

Counter(\(A\))\{

if \(n=1\)\{ // Base case.\\
return (0, \(A\))\\
\} else\{\\
// recursively call counter on 1st half of \(A\) and 2nd half of
\(A\).\\
\((X,x) \leftarrow Counter(A[1, 2, ..., n/2])\) // \(X\) is the number
of inversions in 1st half of \(A\) and \(x\) is the sorted array for
first half.\\
\((Y,y) \leftarrow Counter(A[n/2 +1, n/2 + 2, ..., n])\) // \(Y\) is the
number of inversions in 2nd half of \(A\) and \(y\) is the sorted array
for second half.\\
\((Z,z) \leftarrow merge(x,y)\) // \(Z\) is the number of inversions
between 1st half of \(A\) and 2nd half of \(A\). \(z\) is the sorted
array.

return (\(X+Y+Z\))\\
\}

merge(\(A\), \(B\))\{\\
// \(A\) and \(B\) are sorted arrays.\\
// if \(A[i] > B[j]\), then \(A[i, i+1, ..., n/2] > B[j]\). This means
the number of inversions on \(B[j]\) is equal to the length of
\(A[i, i+1, ..., n/2]\).

count \(\leftarrow 0\)\\
c \(\leftarrow []\)\\
n \(\leftarrow 1\)

while length(\(A\)) \(>0\) or length(\(B\)) \(>0\)\{\\
if \(A[0] > B[0]\)\{\\
remove \(B[0]\) from \(B\)\\
\(c[n] \leftarrow B[0]\)\\
\(n \leftarrow n+1\)\\
count \(\leftarrow\) count \(+\) length(A)\\
\} if \(A[0] < B[0]\)\{\\
remove \(A[0]\) from \(A\)\\
\(c[n] \leftarrow A[0]\)\\
\(n \leftarrow n+1\)\\
\}\\
if length(\(A\)) \(=0\)\{\\
add the rest of \(B\) to \(c\)\\
\}\\
if length(\(B\)) \(=0\)\{\\
add the rest of \(A\) to \(c\)\\
\}\\
return (count,c)\\
\}\}

    The runtime or merge subroutine is \(O(n)\) because the while loop
occurs atmost \(n/2+1\) times and all operations inside the loop is
constant time. There are 2 recursive calls in the Counter algorithm.
Therefore, the runtime should be the same as merge sort. Runtime is
\(O(nlogn)\).

    Counter algorithm proof of correctness by induction:

Input is array \(A[1, 2, ..., n]\) is an unsorted array of length n.\\
Output is the number of inversions in A.

Base case:\\
Suppose \(n=1\). Then, there are no inversions. Therefore, return 0 is
the correct output.

Inductive case:\\
Let \(X\) represent first 1/2 of A, and \(Y\) represent second 1/2 of A.
Assume length of \(A\) is divisible by 2. Assume that the Counter
algorithm returns corrrect number of inversions for input size less than
\(n\). Suppose we call algorithm on input of size \(n\). Then algorithm
will recursively call on \(X\) and \(Y\). By the inductive hypothesis,
these calls will return the correct number of inversions. Since the
merge step works correctly, then the algorithm returns correct number of
inversions after summing up all the inversions from the 2 recursive
calls and the 1 merge step. QED

Merge subroutine proof of correctness:

The Merge subroutine uses the following property: if \(A[0] > B[0]\),
then \(A[0, 1, ..., n/2] > B[0]\). This means the number of inversions
on \(B[0]\) is equal to the length of \(A[0, 1, ..., n/2]\). At each
iteration of the while loop, count is only increased when the property
is satisfied. The while loop runs until A and B are empty, which occurs
atmost \((n/2+1)\) times. After the loop terminates, the counter will
successfully sum up all the iterations between the two halves. QED

    \begin{enumerate}
\def\labelenumi{\arabic{enumi}.}
\setcounter{enumi}{3}
\tightlist
\item
  Recall that when running depth first search on a directed graph, we
  classified edges into 4 categories: tree edges, forward edges, back
  edges, and cross edges. Prove that if a graph is undirected, then any
  depth first search on G will never encounter a cross edge.
\end{enumerate}

    Proof by contradiction:

Assume that \{u,v\} is a cross edge. That means DFS has visited both u
and v, but the edge \{u,v\} is not marked as a tree edge. Since it was
not marked as a tree edge, it means that the edge \{u,v\} was not
visited. Suppose DFS visited u first and not v. DFS would explore all
edges connected to u. Therefore, it would of explored u and then explore
the edge \{u,v\}. This means this edge is a tree edge. This is a
contradiction, which means that \{u, v\} cannot be a cross edge. Both
edges can't be visited at the same time and once either u or v is
visited, \{u,v\} will be explored because the edge is undirected. QED.

    \begin{enumerate}
\def\labelenumi{\arabic{enumi}.}
\setcounter{enumi}{4}
\tightlist
\item
  In the shortest-path algorithm, we are concerned with the total length
  of the path between a source and every other node. Suppose instead
  that we are concerned with the length of the longest edge between the
  source and every node. That is, the bottleneck of a path is defined to
  be the length of the longest edgte in the path. Design an efficient
  algorithm to solve the single source smallest bottleneck problem.
  (find the paths from a source to every other node such that each path
  has the smallest possible bottleneck)
\end{enumerate}

    The proposed algorithm is simply a change on the optimization condition
of Dijkstra's algorithm. Instead of creating paths that have the
smallest distance to \(v\) from \(s\), the proposed algorithm create
paths that have the smallest bottleneck. An array called max is used to
keep track of the max edge weight in a path. For example, \(max[e]\)
would return the max edge weight in the path from \(s\) to \(e\). If a
new path is found where the max edge weight to \(v\) is smaller than the
path currently found, the algorithm will modify the dist array, max
array, and prev array to match the new found path. Everything else, is
the same as Dijkstra's algorithm.

Pseudocode:

Input is \(G=\) Graph(\(E\),\(V\)), \(length[1, 2, ..., n]\), and source
\(s\) such that \(s \in V\). \(length[e]\) is the weight of edge
\(e\).\\
Output is path to \(v\) from \(s\) such that bottleneck is minimized.

\(H \leftarrow \{(s,0), (v, \infty) : v \in V, v \neq s\}\) // H is a
priority heap with size \(n\) where \(n=|V|\).\\
\(dist[s] \leftarrow 0\)\\
\(dist[v] \leftarrow \infty\) for all \(v \neq s\) // runtime is
\(O(n)\) because there are \(n\) vertices.\\
\(prev[v] \leftarrow \emptyset\) for all \(v \in V\) // runtime is
\(O(n)\) because there are \(n\) vertices.

while \(H \neq \emptyset\)\{ \(v \leftarrow\) delete\_min(\(H\)) //
delete\_min gets the minimum value from \(H\). It maintains minimum heap
property. Runtime is \(O(logn)\).\\
For each edge \((v,w) \in E\)\{\\
if \(max[w] >\) max\_value(max{[}\(w\){]}, length(\(v,w\)))\{ //
max\_value returns the max value out of the two inputs.\\
\$max{[}w{]} \textgreater{} \$ max\_value(max{[}\(w\){]},
length(\(v,w\)))\\
\(prev[w] \leftarrow v\)\\
\(dist[w] \leftarrow dist[w] + length(v,w)\) \(Insert(H, w, dist[w])\)
// Maintains minimum heap property. Runtime is \(O(logn)\) \} \} \}

    Because this algorithm only differs from Dijkstra's algorithm on the
optimization condition, the runtime of this algorithm is the same as
Dijkstra's algorithm. Therefore, the runtime is \(O((m+n)logn)\) where
\(m\) is the number of edges and \(n\) is the number of vertices.

    Proof of correctness:\\
Assume Dijkstra's algorithm is correct. The proposed algorithm is a
modification on Dijkstra's algorithm on the optimization condition.
Instead of minimizing the total length of a path to \(v\), this
algorithm minimizes the bottleneck of a path to \(v\). Therefore, it
simply follows from Dijkstra's algorithm that the algorithm will
correctly output paths that minimize the max edge weight in a path.

    \begin{enumerate}
\def\labelenumi{\arabic{enumi}.}
\setcounter{enumi}{5}
\tightlist
\item
  Consider the shortest paths problem in the special case where all edge
  costs are non-negative integers. Describe a modification of Dijkstra's
  algorithm that works in time O(\textbar E\textbar{} +
  \textbar V\textbar M), where M is the maximum cost of any edge in the
  graph.
\end{enumerate}

    Let \(H\) be an array of length \(k|V|\). \(k|V|\) represents the
maximum possible distance from source. \(k\) is the max edge weight in
the graph \(G=(V,E)\). An index \(i\) in \(H\) is empty if there are no
vertices that has distance \(i\) from \(s\). Each index starting from
index 0 represents the distance from \(s\), and each index \(i\) can
hold more than 1 vertex. Initially, s is set to index 0 and every other
vertice is set to \(\infty\). The Insert subroutine is implemented such
a way that it inserts a vertex into a specified index \(i\) in the array
\(H\). Therefore, the runtime on Insert is O(1), constant time. The
Delete\_min subroutine is implemented such that it starts from index 0
in H and delete empty entries from list until encounter a vertex. Each
deletion until encounter vertex will be constant amount of steps and
varies for each iteration. A dist array is initialized to keep track of
all the distances from \(s\) to \(v\). A prev array is used to keep
track of the node that came before node \(v\). A vertex is retrieved
from \(H\) and every adjacent edge of v is explored. Lets say there is
an edge \((v,w)\) from \(v\) that connects to \(w\). If the path to
\(w\) is shorter, then update that path to the shortest distance and
update prev such that the previous node to \(w\) is updated to reflect
the new path. Repeat this process until every edge and vertice is
visited.

Pseudocode:

Input is Graph \(G=(V,E)\), \(length[1, 2, ..., n]\), source
\(s \in V\). \(length[v]\) is the weight of edge \(v\). \(V\) is the set
of all vertices, and \(E\) is the set of all edges.\\
Output is distance to every reachable \(v\) from \(s\).

Let \(H\) match the description used above.\\
\(dist[s] \leftarrow 0\)\\
\(dist[v] \leftarrow \infty\) for all \(v \neq s\)\\
\(prev[v] \leftarrow \emptyset\) for all \(v \in V\)

while \(H \neq \emptyset\)\{\\
\(v \leftarrow\) Delete\_min(\(H\))\\
For each edge \((v,w) \in E\)\{\\
if \(dist[w] > dist[v] + length(v,w)\)\{\\
\(dist[w] \leftarrow dist[v] + length(v,w)\)\\
\(prev[w] \leftarrow v\)\\
\(Insert(H, w, dist[w])\) // Runtime is \(O(1)\)\\
\} \} \}

    The Insert subroutine is implemented such a way that it inserts a vertex
into a specified index \(i\) in the array \(H\). Therefore, the runtime
on Insert is O(1), constant time. The Delete\_min subroutine is
implemented such that it starts from index 0 in H and delete empty
entries from list until encounter a vertex. Each deletion until
encounter vertex will take a constant amount of steps and varies for
each iteration. However, we know that each vertice is visited 1 time and
the total length of \(H\) is \(k|V|\). Therefore, we know that the total
amount of steps that Delete\_min takes after the algorithm terminates is
\(k|V|\). Therefore, Delete\_min has a total runtime of \(O(k|V|)\) when
algorithm terminates. Formally, let \(X_{1}, X_{2}, ..., X_{k|V|}\) be
the number of steps of Delete\_min for each iteration of the while loop.
It is true that \(\sum_{i=1}^{k|V|} X_{i} = k|V|\) and each Insert call
has runtime \(O(1)\).

\([O(1) + O(1), ... + O(1)] + [X_{1}, X_{2}, ..., X_{k|V|}] = |E|O(1) + k|V| = O(|E| + k|V|)\)

    Proof of correctness:

Assume that Dijkstra's algorithm is correct. The only change in the
proposed algorithm is that the implemented data structure is different.
The Insert subroutine inserts vertices into index that correspond to the
distance of vertex \(v\) from \(s\). This ensures that vertices are
ordered from smallest distance to highest distance like in a binary
heap. The Delete\_min operation always return minimum vertex distance,
which is the same as in Dijkstra's implementation. Therefore, it follows
that the proposed algorithm is correct. QED

    \begin{enumerate}
\def\labelenumi{\arabic{enumi}.}
\setcounter{enumi}{6}
\tightlist
\item
  The risk-free currency exchange problem offers a risk-free way to make
  money. Suppose we have currencies \(c_{1}, c_{2}, ... c_{n}\). For
  every two currencies \(c_{i}\) and \(c_{j}\), there is an exchange
  rate \(r_{i,j}\) such that you can exchange one unit of \(c_{i}\) for
  \(r_{i,j}\) units of \(c_{j}\). Note that if
  \(r_{i,j} \cdot r_{j, i} > 1\), then you can make money simply by
  trading units of currency \(i\) into units of currency \(j\) and back
  again. This almost never happens, but occasionally (because the
  updates for exchange rates do not happen quickly enough) for very
  short periods of time exchange traders can find a sequence of trades
  that can make risk-free money. That is, if there is a sequence of
  currencies \(c_{i_{1}}, c_{i_{2}}, ..., c_{i_{k}}\) such that
  \(r_{i_{1}, i_{2}} \cdot r_{i_{2}, i_{3}} \cdot... \cdot r_{i_{k-1}, i_{k}} \cdot r_{i_{k}, i_{1}} > 1\),
  then trading one unit of \(c_{i_{1}}\) into \(c_{i_{2}}\) and trading
  that into \(c_{i_{3}}\) and so on will yield a profit. Design an
  efficient algorithm to detect if a risk-free currency exchange exists.
  (Need not actually find the path, just Yes or No)
\end{enumerate}

    Have \(c_{1}, c_{2}, ... c_{n}\) make up the vertices of a graph where
\(r_{c_{1}, c_{2}}, r_{c_{2}, c_{3}}, ..., r_{c_{n-1}, c_{n}}\) are edge
weights for corresponding edges. Do a \(-log(x)\) transformation on all
edge weights, and then call Bellman\_ford on the graph with transformed
edge weights.

Pseudocode:

Input: Graph \(G=(V,E)\), where \(c_{1}, c_{2}, ... c_{n}\) make up the
vertices. Let the following
\(r_{c_{1}, c_{2}}, r_{c_{2}, c_{3}}, ..., r_{c_{n-1}, c_{n}}\) be edge
weights for the corresponding edge, and let length be the array that
contains the edge weights. So \(length[(c_{1}, c_{2})]\) contains the
edge weight \(r_{c_{1}, c_{2}}\).\\
Output: Determines if there is a risk-free currency exchange or no there
isn't.

for edge \(\in\) E\{\\
\(length[edge] \leftarrow -log(length[edge])\) // Doing a negative log
transformation for every edge. Runtime is \(O(m)\) because there are
\(m\) edges.\\
\}

Bellman\_Ford(\(G=(V,E)\), length) // Bellman\_Ford will output `Yes' if
a negative cycle, risk-free currency exchange, exists

    The runtime is \(O(|E|) + O(|E||V|) = O(|E||V|)\) because clearly
\(|E||V|\) term dominates. The Bellman\_ford algorithm takes up the most
time.

    Proof of correctness:

\(r_{i_{1}, i_{2}} \cdot r_{i_{2}, i_{3}} \cdot... \cdot r_{i_{k-1}, i_{k}} \cdot r_{i_{k}, i_{1}} > 1\)\\
\(=log(r_{i_{1}, i_{2}} \cdot r_{i_{2}, i_{3}} \cdot... \cdot r_{i_{k-1}, i_{k}} \cdot r_{i_{k}, i_{1}}) > log(1)\)\\
\(=log(r_{i_{1}, i_{2}} \cdot r_{i_{2}, i_{3}} \cdot... \cdot r_{i_{k-1}, i_{k}} \cdot r_{i_{k}, i_{1}}) > 0\)\\
\(=-log(r_{i_{1}, i_{2}} \cdot r_{i_{2}, i_{3}} \cdot... \cdot r_{i_{k-1}, i_{k}} \cdot r_{i_{k}, i_{1}}) < 0\)\\
\(=-log(r_{i_{1}, i_{2}})-log(r_{i_{2}, i_{3}})... -log(r_{i_{k-1}, i_{k}})-log(r_{i_{k}, i_{1}}) < 0\)

Assume Bellman\_ford algorithm is correct. Clearly, the proposed
algorithm is correct because it only does a \(-log(x)\) transformation
on all the edge weights and then calls Bellman\_ford algorithm.
Transforming every edge weight allows us turn the condition for
risk-free currency exchange into a condition that also matches the
condition for negative weight cycles. Therefore, Bellman\_Ford algorithm
is able to detect risk-free curency exchange. QED


    % Add a bibliography block to the postdoc
    
    
    
    \end{document}
